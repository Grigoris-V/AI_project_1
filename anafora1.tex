% !TEX encoding = UTF-8 Unicode
\documentclass[12pt, a4 paper]{article}
\usepackage{graphicx}


\usepackage{amsmath}
\usepackage{amssymb}
\usepackage{fancyhdr}

\usepackage[english,greek]{babel}
\usepackage{ucs}
\usepackage[utf8x]{inputenc}
%\usepackage[latin1]{inputenc}
\usepackage[T1]{fontenc}
\usepackage{amsfonts}

\usepackage{pgf}
\usepackage{tikz}
\usetikzlibrary{arrows,automata}

\oddsidemargin0cm
\topmargin-2cm     %I recommend adding these three lines to increase the 
\textwidth16.5cm   %amount of usable space on the page (and save trees)
\textheight23.5cm  

\newcommand{\question}[2] {\vspace{.25in} \hrule\vspace{0.5em}
\noindent{\bf #1: #2} \vspace{0.5em}
\hrule \vspace{.10in}}
\renewcommand{\part}[1] {\vspace{.10in} {\bf (#1)}}

\usepackage{algorithm}
\usepackage{algpseudocode}
\usepackage{pifont}
\algrenewcommand\textproc{} %Gia na exoun mikra ta procedures

\usepackage{fancyhdr}
\pagestyle{fancy}
\fancyhf{}
\fancyfoot[CO, CE]{\thepage}

 
 \begin{document}

\begin{titlepage}
	\begin{center}
	 
	\line(1,0){300}\\
	[0.25in]
	\huge{\bfseries{Τεχνητή Νοημοσύνη\\
	1\textsuperscript{ο} Θέμα}}\\
	[2mm]
	\line(1,0){300}\\
	[1cm]
	\textit{Βελέγκας Γρηγόριος}\\
	\textit{Α.Μ. 03113011}\\
	[5mm]
	\textit{Λωτίδης Κυριάκος}\\
	\textit{Α.Μ. 03113033}\\
	[10mm]
	\textit{ΗΜΜΥ ΕΜΠ}\\
	\textit{5/12/2016}\\
	\end{center}
	
\end{titlepage}

\newpage

\setcounter{page}{1}
Για το διάβασμα των δεδομένων από τα αρχεία χρησιμοποιήθηκαν \textlatin{bufferedReaders} και για κάθε γραμμή γίνεται ένα \textlatin{parsing} ώστε να πάρουμε μεμονωμένα τα στοιχεία που μας ενδιαφέρουν. Δημιουργήθηκε μια κλάση \textlatin{Node} για την αναπαράσταση των κόμβων του γράφου, που περιέχει τις συντεταγμένες του κάθε σημείου σε μορφή \textlatin{String} και \textlatin{Double}, αφού και οι δύο τύποι ήταν χρήσιμοι για διαφορετικές πράξεις, μια λίστα με τους γείτονές του και την εκάστοτε απόσταση (σε μορφή \textlatin{ArrayList}), έναν δείκτη προς τον πατέρα του στο μονοπάτι που δημιουργείται από τον Α*, καθώς και πεδία τύπου \textlatin{Double} για τις τιμές \textlatin{real} και \textlatin{heuristic} των αποστάσεων. Ο γράφος υλοποιείται με ένα \textlatin{Hashtable} που περιέχει τους κόμβους που αναφέρθηκαν πριν. Η επιλογή του \textlatin{Hashtable} έγινε επειδή προσφέρει αναζήτηση για τυχαίο στοιχείο σε χρόνο $\Theta(1)$, κάτι που ήταν πολύ χρήσιμο αφού κατά το διάβασμα των συντεταγμένων των σημείων θέλαμε να κάνουμε γρήγορο έλεγχο για το κατά πόσο υπάρχει ήδη ένας κόμβος, αφού με αυτόν τον τρόπο δηλώνονται οι διασταυρώσεις. Για τον Α* χρησιμοποιήθηκε ένα \textlatin{HashMap} που περιέχει τους κόμβους που έχουμε ήδη επισκεφθεί έτσι ώστε όταν τους ξανασυναντήσουμε να μην τους επεκτείνουμε (η επιλογή της δομής έγινε και πάλι με κριτήριο τη γρήγορη αναζήτηση στοιχείου), μια \textlatin{PriorityQueue} που μοντελοποιεί το μέτωπο αναζήτησης (με \textlatin{comparator} το άθροισμα της ευριστικής και της πραγματικής απόστασης) καθώς και άλλο ένα \textlatin{HashMap} που περιέχει επίσης τους κόμβους που βρίσκονται στο μέτωπο αναζήτησης, αφού όταν συναντήσουμε έναν κόμβο θέλουμε να εξετάσουμε κατά πόσο υπάρχει ήδη στο μέτωπο. Σε περίπτωση που υπάρχει και η νέα του \textlatin{real} απόσταση είναι μικρότερη από την προηγούμενη πρέπει να του αλλάξουμε την προτεραιότητα και ο μόνος τρόπος για να το κάνουμε αυτό είναι με την εξαγωγή και έπειτα την εισαγωγή του. Η αναζήτηση στοιχείου σε μια \textlatin{PriorityQueue} της \textlatin{Java} χρειάζεται γραμμικό χρόνο, οπότε χρησιμοποιώντας το \textlatin{HashMap} έχουμε σημαντική βελτίωση στο χρόνο αναζήτησης, έχοντας όμως κάποιο επιπλέον κόστος στη μνήμη. Η \textlatin{real}  απόσταση κάθε κόμβου είναι το άθροισμα της μέχρι τώρα διαδρομής από το αρχικό σημείο, δηλαδή το εκάστοτε ταξί, και η \textlatin{heuristic} είναι η ευκλείδια απόσταση μέχρι τον πελάτη, η οποία αποτελεί υποεκτίμηση της πραγματικής απόστασης οπότε είμαστε βέβαιοι ότι ο αλγόριθμος θα βρει τη βέλτιστη λύση.\\
Οι συντεταγμένες του πελάτη καθώς και τον ταξί αποθηκεύονται επίσης σε ένα αντικείμενο τύπου \textlatin{Node}. Το σύνολο των ταξί αποθηκεύεται σε ένα \textlatin{HashMap} με κλειδί το προαναφερθέν \textlatin{Node} και τιμή τον κωδικό του κάθε ταξί. Τα σημεία από το αρχείο \textlatin{nodes} αποθηκεύονται στην ίδια δομή και η εύρεση των γειτόνων τους γίνεται με την εξής διαδικασία: Διαβάζουμε τις συντεταγμένες κάθε γραμμής και κρατάμε τον κωδικό της οδού. Ελέγχουμε κατά πόσο υπάρχει ήδη το σημείο στο γράφο, αφού στην περίπτωση αυτή έχουμε διασταύρωση και δε χρειάζεται να το τοποθετήσουμε εκ νέου. Αν ο κωδικός οδού του τρέχοντος σημείου είναι ίδιος με του προηγούμενου προσθέτουμε μια ακμή, δηλαδή έναν κόμβο στη λιστα γειτόνων του κάθε σημείου, διαφορετικά δεν προσθέτουμε κάποια σύνδεση. Από τα δεδομένα που μας δόθηκαν δε χρειάστηκε να κρατήσουμε αποθηκευμένα ούτε τον κωδικό της οδού ούτε και το όνομά της. Επειδή οι συντεταγμένες των ταξί και του πελάτη δεν ήταν σημεία του χάρτη επιλέξαμε να προσθέσουμε μια ακμή μεταξύ του καθενός απ' αυτά με τον κοντινότερό του κόμβο ώστε να μπορέσουμε να κάνουμε τον γράφο συνεκτικό.\\
Για την εκτέλεση του προγράμματος, τα ονόματα των αρχείων δίνονται στον κώδικα και τα αποτελέσματα γράφονται στο \textlatin{standard output}. Για να τα δούμε στο αρχείο \textlatin{kml} τα αντιγράφουμε στις κατάλληλες θέσεις.\\
Στα αρχεία \textlatin{my\_client\_athens.csv, my\_taxis\_athens.csv} φαίνονται κάποιες επιπλέον δοκιμές για τον δοθέντα χάρτη. Έγιναν επιπλέον πειραματισμοί με τον χάρτη της Θεσσαλονικής, ο οποίος βρίσκεται στο αρχείο \textlatin{thessaloniki.csv}, και τα αρχεία των ταξί και του πελάτη είναι τα \textlatin{thess\_taxis.csv} και \textlatin{thess\_client.csv} αντίστοιχα. Το αρχείο \textlatin{athina\_default.kml} είναι για τα δεδομένα που μας δόθηκαν, το \textlatin{athina\_sample.kml} είναι το αποτέλεσμα του δικού μας πειραματισμού στο χάρτη της Αθήνας ενώ το αρχείο \textlatin{thessaloniki.kml} είναι το αποτέλεσμα της εκτέλεσης στο χάρτη της Θεσσαλονίκης. Στο φάκελο \textlatin{src} βρίσκεται ο πηγαίος κώδικας καθώς και τα διάφορα αρχεία που χρειάζονται. Η \textlatin{main} βρίσκεται στην κλάση \textlatin{Make}.\\
[10mm]
Στη συνέχεια παρατίθενται εικόνες από τα αποτελέσματα του προγράμματός μας για τις εκάστοτε εκτελέσεις.\\
$\bullet$ Αθήνα-Δοθέντα δεδομένα:\\
[4mm]
\includegraphics[scale=0.65]{athina_default}\\
[4mm]
\newpage
$\bullet$ Αθήνα-Επιπλέον δεδομένα (όπως βλέπουμε οι διαδρομές κάποιων ταξί ταυτίζονται σε κάποιο σημείο γι' αυτό δεν απεικονίζονται όλα):\\
[4mm]
\includegraphics[scale=0.5]{athina_sample}\\
[4mm]
$\bullet$ Θεσσαλονίκη:\\
[4mm]
\includegraphics[scale=0.5]{thessaloniki}\\

 \end{document}